\documentclass[conference]{IEEEtran}
\IEEEoverridecommandlockouts

\begin{document}

\title{Ensuring Fairness with Transparent Auditing of Ethical Bias in AI Systems\\
\thanks{Identify applicable funding agency here. If none, delete this.}
}

\author{\IEEEauthorblockN{1\textsuperscript{st} Given Name Surname}
\IEEEauthorblockA{\textit{dept. name of organization (of Aff.)} \\
\textit{name of organization (of Aff.)}\\
City, Country \\
email address or ORCID}
\and
\IEEEauthorblockN{2\textsuperscript{nd} Given Name Surname}
\IEEEauthorblockA{\textit{dept. name of organization (of Aff.)} \\
\textit{name of organization (of Aff.)}\\
City, Country \\
email address or ORCID}
}

\maketitle

\begin{abstract}
With the rapid advancement of AI, there is a growing trend to integrate AI into decision-making processes. However, AI systems may exhibit biases that lead decision makers to draw unfair conclusions. Notably, the COMPAS system used in the American justice system to evaluate recidivism was found to favor racial majority groups; specifically, it violates a fairness standard called equalized odds. Various measures have been proposed to assess AI fairness. We present a framework for auditing AI fairness, involving third-party auditors and AI system provider, and we have created a tool to facilitate systematic examination of AI systems. Inclusion of independent and trusted auditors is pivotal for objectivity and accountability; third parties are often further necessary for their specialized expertise in relevant domains. Auditors, equipped with our tool, can thoroughly review AI systems for bias and fairness violations. The tool is open-sourced and publicly available. Unlike traditional AI systems, we advocate a transparent white-box and statistics-based approach. It can be utilized by third party auditors, AI developers, or the general public for reference when judging the fairness criterion of AI systems.
\end{abstract}

\section{Introduction}
In recent years, the accelerating pace of artificial intelligence (AI) technologies have revolutionized numerous fields. In healthcare, AI-driven diagnostic systems streamline disease identification, while in finance, automated trading algorithms analyze market trends to execute optimal trades swiftly. This remarkable progress has reached diverse industries,sprouting out various applications for different purposes.

One of its most profound impact is how AI has transformed decision-making processes across sectors. By harnessing vast amounts of data and employing advanced algorithms, AI empowers organizations to make more informed and strategic decisions. From assessing job applicants to determining school admissions, AI-driven insights offer efficient and analytical advantages.




\section{Ease of Use}
bias sources example (COMPAS)
fairness is subjective, based on culture and often not quantitative
for AI system we use quantitative fairness measures
transparency and objectivity example : COMPAS
third party needed for auditing example : propublica like entity can use our stuff; we're not replacing but supplement; domain knowledge
we propose a framework

\end{document}




