\documentclass[conference]{IEEEtran}
\IEEEoverridecommandlockouts

\begin{document}

\title{Ensuring Fairness with Transparent Auditing of Ethical Bias in AI Systems\\
\thanks{Identify applicable funding agency here. If none, delete this.}
}

\author{\IEEEauthorblockN{1\textsuperscript{st} Given Name Surname}
\IEEEauthorblockA{\textit{dept. name of organization (of Aff.)} \\
\textit{name of organization (of Aff.)}\\
City, Country \\
email address or ORCID}
\and
\IEEEauthorblockN{2\textsuperscript{nd} Given Name Surname}
\IEEEauthorblockA{\textit{dept. name of organization (of Aff.)} \\
\textit{name of organization (of Aff.)}\\
City, Country \\
email address or ORCID}
}

\maketitle

\begin{abstract}
With the rapid advancement in the field of AI, there is a growing trend
to integrate AI into the information flow of the decision making
processes. However, the deployment of AI algorithms in this respect
raises ethical considerations, particularly, regarding fairness. AI
systems may exhibit biases, caused by various sources, that can
potentially lead decision makers to derive unfair conclusions. For
example, in 2016, journalists at ProPublica discovered that COMPAS, the
algorithm system used by the American justice system to evaluate
recidivism, displayed unfair treatment to different racial groups,
favoring the majority groups while harming the minority groups;
specifically, it violates a fairness measure called equalized odds. In
recent years, researchers have devised a number of measures designed to
evaluate the fairness of AI systems. We propose here a framework for
auditing fairness of AI systems by including a third party auditor and
an AI system provider and create a tool to facilitate the examining of
such systems. It's pivotal for the integrity of the audits that the
framework includes an independent and trusted arbiter for their
objectivity and accountability; third party auditors are often further
necessary for their specialized expertise in the relevant domains such
as medicine and law. Auditors, equipped with our tool, can thoroughly
review the AI systems for bias and fairness violations. The tool is
open-sourced, easily accessible, and publicly available. Unlike
traditional AI systems, we advocate a transparent white-box and
statistics based approach. It can be utilized by third party auditors,
model makers, or the general public as a reference point when judging
the fairness criterion of AI systems.
\end{abstract}

\begin{IEEEkeywords}
keyword, keyword
\end{IEEEkeywords}

\end{document}




